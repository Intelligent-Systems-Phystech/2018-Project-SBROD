\documentclass[12pt,twoside]{article}
\usepackage{jmlda}
\begin{document}
\title{Задачи оптимизации, сочетающей классификацию и регрессию, для оценки энергии связывания белка и маленьких молекул}
\author{Noskova~E.\,S., Kachkov~S.\,S., Sidorenko~A.\,A.}

\abstract{Задача оценки качества белка актуальна на сегодняшний день благодаря своей применимости в прогнозировании структуры белка - фундаментальной и все же открытой проблемы в структурной биоинформатике. Данная статья посвящена новому методу оценки качества белка(SBROD), использующему только форму белкового скелета.
Предлагаемый метод выводит свою скоринговую функцию, основываясь на учебном наборе белковых моделей. Функция подсчета SBROD состоит из четырех терминов, связанных с различными структурными особенностями белка.}
\maketitle
\section{Введение}
Белки играют важную роль в таких фундаментальных биологических процессах, как биологический перенос, образование новых молекул или клеточная защита. Это значение вызвало широкое исследование их свойств, которое требует дорогостоящих экспериментов. Потенциально они могут быть заменены более дешевыми и более быстрыми численными методами моделирования. Большинство предложенных методов для прогнозирования структуры белка сначала генерируют набор правдоподобных белковых моделей, а затем ранжирует их с использованием определенного QA метода. Обычно эти методы основаны на скоринговых функциях, которые предсказывают сходство между белковыми моделями и целевыми структурами в таких показателях сходства, как RMSD, GDT-TS и TM-score. В частности, RMSD измеряет среднее расстояние между атомами двух наложенных конформаций белка. GDT-TS и TM-score разработаны для оценки качества белковых моделей, являющихся независимыми от размера белка, и устойчивые к локальным структурным ошибкам. Существует два основных метода оценки качества. Модель консенсуса определяят качество отдельных моделей белка на основе их статистики в оценочном наборе. Напротив, одномодельные методы рассматривают только атомы оцениваемого белка без дополнительной информации о других моделях, и, следовательно, их можно использовать для конформационной выборки. Среди недавно предложенных одномодельных QA методов существуют два основных подхода к разработке оценочной функции: основанный на физической модели и основанный на данных. Построенные на физической модели функции подсчета очков используют некоторую информацию о взаимодействиях в системе, например, принцип минимизации энергии Гиббса. Однако точная оценка свободной энергии Гиббса требует выборки огромного количества конформационных состояний, которые в большинстве практических случаев является трудноразрешимым. Данные методы направлены на построение скоринговых функций, которые аппроксимируют энтальпическую часть свободной энергии Гиббса, раскладывая её на сумму добавочных членов (вкладов), которые представляют собой растяжение связей или углов, диэдральные потенциалы, электростатические и Ван дер Ваальсовые взаимодействия и т. д. Наряду с физическими подходами, существуют подходы, основанные на данных, которые выводят энергию молекулярных взаимодействий из баз данных, предполагающих определенное распределение конформаций или минимизируя определенную функцию потерь. Соответствующие функции подсчета очков обычно производятся либо путем машинного обучения, либо путем оценки вероятности некоторых конформаций (статистические методы) с использованием статистических данных об определенных белковых структурах из структурных баз данных. В данной работе изучается новый метод оценки качества белка, Smooth Backbone-Reliant Orientation-Dependent (SBROD). SBROD является одномодельным методом QA, который оценивает белок модели, используя геометрические структурные особенности. Он требует только координат белковой основы и, следовательно, нечувствителен к конформациям боковых цепей. Кроме того, SBROD функция скоринга непрерывна по отношению к координатам атомов белка, что делает её также потенциально применимой для использования в молекулярной механике.
\bigskip
\nocite{*}
\bibliographystyle{unsrt}
\bibliography{references}
\end{document}

