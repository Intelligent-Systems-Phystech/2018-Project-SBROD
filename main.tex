\documentclass[12pt,twoside]{article}
\usepackage{jmlda}
\begin{document}
\title{Задачи оптимизации, сочетающей классификацию и регрессию, для оценки энергии связывания белка и маленьких молекул}
\author{Noskova~E.\,S., Kachkov~S.\,S., Sidorenko~A.\,A.}

\abstract{Задача оценки качества белка актуальна на сегодняшний день благодаря своей применимости в прогнозировании структуры белка - фундаментальной и все же открытой проблемы в структурной биоинформатике. Данная статья посвящена расширению метода оценки качества белка(SBROD), использующему для своих вычислений только форму белкового скелета и переобучение его на более надежных метриках CAD и LDDT. А также улучшению метода посредством изменения функции потерь, используя Z-score.}
\maketitle
\section{Введение}
Белки играют важную роль в таких фундаментальных биологических процессах, как биологический перенос, образование новых молекул или клеточная защита. Именно поэтому задача оценки качества моделей белковой структуры так важна на сегодняшний день. Существует множество методов для решения данной задачи, но не все из них достаточно оптимальны. Интересным является метод SBROD, предложенный учеными Михаилом Карасиковым, Гийом Пажем и Сергеем Грудининым, который производит оценку качества белковой модели, основываясь только на геометрии белка. В нашей работе мы хотим улучшить представленный метод путем переобучения модели на метриках CAD и lDDT.
CAD метрика количественно определяет различия между физическими контактами в модели и в эталонной структуре. В ней используется понятие разности контактных площадей остаток-остаток, введенное Абагяном и Тотровым. Контактные области, лежащие в основе оценки, получены с использованием тесселяции структуры белка Вороного. Локальный тест разности расстояний (lDDT)-это метрика без суперпозиции, которая оценивает локальные разности расстояний всех атомов в модели, включая проверку стереохимической правдоподобности. LDDT хорошо подходит для оценки качества локальной модели, сохраняя хорошую корреляцию с глобальными метрика ми.
Выбор этих метрик обуславливается их приемуществом перед другими, ранее используемыми. Также предполагается улучшение функции потерь для оптимального выбора лучшей модели. Для этого будет применена оптимизация Z-score, которая представляет из себя числовое измерение отношения значения к среднему значению в группе значений. Если Z-оценка равна 0, она представляет собой оценку, идентичную средней. Z-оценки также могут быть положительными или отрицательными, с положительным значением, указывающим, что оценка выше среднего, а отрицательный показатель, указывающий, что он ниже среднего.
\bigskip
\nocite{*}
\bibliographystyle{unsrt}
\bibliography{references}
\end{document}

